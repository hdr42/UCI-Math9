
\documentclass[addpoints,12pt]{exam}
\usepackage{amssymb,amsmath,amsthm,graphicx}
\usepackage{tikz}
\usepackage{listings}
\usepackage{courier}
\usepackage{graphicx}
\usepackage{scrextend}

\lstset{frame=l,xleftmargin=\fboxsep,xrightmargin=-\fboxsep,colframe=gray}
\lstset{basicstyle=\ttfamily\footnotesize,breaklines=true}


\newcommand{\code}[1]{{\texttt{#1}}}
\newcommand{\mcode}[1]{{\text{\texttt{#1}}}}

\linespread{1.2}
\usepackage{color}
\definecolor{gray}{rgb}{0.3,0.3,0.3}



\pagestyle{headandfoot}
\runningheadrule
\firstpageheader{Math 9}{\,}{Sample Midterm}
\runningheader{Math 3A}
              {\,}
              {Midterm ver A, Page \thepage\ of \numpages}
\firstpagefooter{}{}{}
\runningfooter{}{}{}
\newtheorem{theorem}{Theorem}
\newtheorem{definition}{Definition}
\newtheorem{expectation}{Expectation}

\newcommand{\RR}{\mathbb{R}}

\begin{document}
\begin{center}
\fbox{\fbox{\parbox{5.5in}{\centering
{\tt Directions:} The exam is 50 minutes long. Please read each question carefully. 
\vspace{10pt}

{\sc each question is worth 20 points}
When asked to write code, you should write working Python code that has correct syntax. 

Use the backs of the pages if needed.
}}}
\end{center}


\vspace{0.2in}

\makebox[\textwidth]{Last Name:\enspace\hrulefill}

\vspace{0.2in}

\makebox[\textwidth]{First Name:\enspace\hrulefill}

\vspace{0.2in}

\makebox[\textwidth]{Student ID \#:\enspace\hrulefill}

\vspace{0.2in}

\vspace{1in}

\gradetable

\newpage

\begin{questions}

\question[20]
Write down the output of the following programs.

\begin{enumerate}
  \item 
\begin{lstlisting}[language=python]
i = 5 
while i < 100:
  print(i)
  i *= 2 
print(i)
\end{lstlisting}
    \vfill

\item 
\begin{lstlisting}[language=python]
def f(n):
    for i in range(10):
        if i > n:
            continue
        n = n // 2
    return n

print(f(256))
\end{lstlisting}

    \vfill

  \item 
\begin{lstlisting}[language=python]
def g(n):
  if n == 0:
    return [] 
  return [n % 2] + g(n // 2) 

print(g(255))
\end{lstlisting}

    \vfill
\end{enumerate}

\newpage
\question[20] Produce the following lists using list comprehension:
\begin{enumerate}
  \item \code{[1,1,1,1,2,2,2,2,3,3,3,3,4,4,4,4]}
    \vfill
  \item \code{[1,-1,2,-2,3,-3,4,-4,5,-5]}
    \vfill
  \item \code{[12,10,8,6,4,2,0,0,0,0]}
    \vfill
\end{enumerate}

\newpage
\question[20] Write down the output of the following code:

\begin{enumerate}
  \item (10 pts) 
\begin{lstlisting}[language=python]
reduce(lambda x, y: x*y, range(1,7))
\end{lstlisting}
\vfill
  \item (10 pts)
    \begin{lstlisting}[language=python] 
reduce(lambda a,d: 10*a+d, [1,2,3,4,5,6,7,8])
\end{lstlisting}
\vfill

\end{enumerate}


\newpage
\question[20] Write down a Python function \code{spread(xs)} that will return the difference between the largest and smallest element of a list \code{xs}. (e.g. \code{spread([1,2,-1,5,10,12,0,0,4])} is \code{13}) (To receive full marks, you need to write a little explanation of which part of your code does what)


\newpage
\question[20] Write down a Pyhon function \code{gcd(x,y)} that returns the greatest common divisor of two positive integers \code{x} and \code{y}. (To receive full marks, you need to write a little explanation of which part of your code does what) 
\end{questions}
\end{document}
